\documentclass[12pt]{article}
\usepackage[margin=1in]{geometry}
\usepackage{amsthm, amsmath, amsfonts, amssymb}
\usepackage{graphicx}
\usepackage[shortlabels]{enumitem}
\usepackage{setspace}
\usepackage{tcolorbox}
\usepackage[
  lambda,
  advantage,
  operators,
  adversary,
  landau,
  probability,
  sets,
  % notions,
  % logic,
  % ff,
  % mm,
  % primitives,
  % oracles,
  events,
  % complexity,
  asymptotics,
  keys,
]{cryptocode}
\doublespacing

%%% cryptocode tweaks
% Multi-letter variables, e.g., "ctr"
\NewDocumentCommand\var{m}{\ensuremath{\mathit{#1}}}
% Make keys and polynomials look like other identifiers.
% (No idea why cryptocode treats them specially.)
\RenewDocumentCommand\pckeystyle{m}{\ensuremath{\var{#1}}}
\RenewDocumentCommand\pcpolynomialstyle{m}{\ensuremath{\var{#1}}}
% Add 1pt padding to \gamechange (based on original definition).
\renewcommand{\gamechange}[2][gamechangecolor]{%
  {\setlength{\fboxsep}{1pt}%
    \colorbox{#1}{#2}}}
% Shorthand for \pcalgostyle
\NewDocumentCommand\algo{}{\pcalgostyle}
% Oracles
\NewDocumentCommand{\mathsc}{m}{{\normalfont\textsc{#1}}}
\RenewDocumentCommand\pcoraclestyle{m}{\ensuremath{\mathsc{#1}}}
\NewDocumentCommand\oracle{m}{\pcoraclestyle{#1}}
% Games
\newcommandx{\game}[4][3=\adv,4=(\secpar)]{{\operatorname{#1}_{#2}^{#3}#4}}
%\newcommand{\Game}{\algo{Game}}
% Misc
\newcommand{\pcsc}{\,;~}

\newcommand{\N}[0]{\mathbb{N}}
\newcommand{\Z}[0]{\mathbb{Z}}
\newcommand{\Q}[0]{\mathbb{Q}}
\newcommand{\F}[0]{\mathbb{F}}
\newcommand{\G}[0]{\mathbb{G}}
\newcommand{\pr}[1]{\text{Pr}\left[#1\right]}
\newcommand{\overarrow}[1]{\stackrel{#1}{\rightarrow}}
\newcommand{\verts}[1]{\left\vert #1\right\vert}
\newcommand{\M}{\mathcal{M}}
\newcommand{\C}{\mathcal{C}}
\newcommand{\K}{\mathcal{K}}
\newcommand{\Kpriv}{\mathcal{K}_{\text{priv}}}
\newcommand{\Kpub}{\mathcal{K}_{\text{pub}}}

\newtheorem{claim}{Claim}
\newtheorem{conj}{Conjecture}
\newtheorem{question}{Question}
\newtheorem{exercise}{Exercise}
\newtheorem{reading}{Reading}
\newtheorem{thm}[claim]{Theorem}
\newtheorem{lemma}[claim]{Lemma}
\newtheorem{prop}[claim]{Proposition}
\newtheorem{cor}[claim]{Corollary}
\theoremstyle{definition}
\newtheorem{definition}{Definition}

\theoremstyle{remark}
\newtheorem*{rem}{Remark}

\theoremstyle{definition}
\newtheorem{example}{Example}[section]


\begin{document}
\title{Week 6}
\author{Nadav Kohen}
\date{July 30, 2025}
\maketitle

Last week we looked at some advanced topics: Sigma protocols and the OMDL and AMODL assumptions. In our final week of prework, we will review and reinforce our understanding of the Schnorr security proof, as well as practice writing security definitions for new protocols. We will also briefly introduce Schnorr multi-signatures and Schnorr threshold signatures.\\

Before we begin, let us take a moment to take another pass at the range proofs that were discussed in our last call together.

\begin{exercise} Recall that for a group $\G$ of worder $p$ with generators  $g$ and $h$, a Pedersen commitment to $x\in\Z_p$ with randomness $y\in\Z_p$ is of the form $C = g^xh^y$.
\begin{enumerate}[(a)]
\item Construct a special HVZK Sigma protocol with special soundness for the statement that a commitment value, $C$, is a Pedersen commitment for the value $x = 1$ (without revealing $y$, of course). Create this protocol by starting with the Schnorr ID protocol for $h^y$ and modifying the verification step at the end.
\item Next, use the OR construction for Sigma protocols from last week to construct a special HVZK Sigma protocol with special soundness for the statement that a commitment value, C, is a Pedersen commitment for either the value $x = 0$, or the value $x = 1$ (without revealing which, of course).
\item Instead of computing $C = \text{Commit}(x; y)$ by generating a random value of $y$ and returning $g^xh^y$, we will instead generate $n$ random values, $y0, \ldots, y_{n-1}$ and let $y$ be equal to $y_0 + 2y_1 + \cdots + 2^{n-1}y_{n-1}\bmod p$. Then, if $x$ is equal to $x_0 + 2x_1 + \cdots + 2^{n-1}x_{n-1}$ (where the $x_i$ are $x$’s bits in binary) we get that $C = \prod_i \text{Commit}(x_i; y_i)(2^i)$ and each of the $x_i$ is either equal to 0 or 1! Because our commitments are perfectly hiding, it is safe to share the values $\text{Commit}(x_i; y_i)$.

Using your solution to the previous part, write down an explicit Sigma protocol for the statement that all of the commitments, $C_i = \text{Commit}(x_i; y_i)$, are committed to a bit (0 or 1). Alongside the verification that $C = \prod_i C_i^{2^i}$, this constitutes a range proof!
\end{enumerate}
\end{exercise}

Moving on, as we have now seen more than a couple times, reducing the security of protocols such as Schnorr to underlying computational assumptions usually (outside of the AGM) relies on the use of ``rewinding''/``forking'' arguments, where an adversary is run multiple times with related inputs and a fixed source of randomness. Because of the prevalence of these kinds of arguments, security proofs can oftentimes be simplified by refactoring so that there is a single independent forking lemma beforehand, which is invoked within a security proof.\\

I actually chose our first Schnorr security proof to be out of the Katz and Lindell book, in part, because they do not do this. But now that we have the proper motivation,

\begin{reading}\label{forking}
Read about a simple and general forking lemma in:\\
Mihir Bellare and Gregory Neven - New Multi-Signature Schemes and a General Forking Lemma - Section 3. DO NOT (YET) READ SECTION 4!
\end{reading}

I recommend that, on a first reading, you should not worry too much about the details of the proof of the lemma, and instead focus primarily on understanding what the lemma is claiming.

\begin{exercise}
Using the lemma from Reading \ref{forking}, write up a proof in your own words that Schnorr digital signatures are secure in the Random Oracle Model if the Discrete Log problem is hard. You may refer back to the proofs in Katz and Lindell, but I would like a direct proof that does not explicitly invoke the Fiat-Shamir transform. I encourage you to use this opportunity to be very pedantic and to ensure that you are comfortable with all of the details!
\end{exercise}

Once you have completed, or at least nearly completed, this exercise,

\begin{reading}\label{Schnorr}
Read the general forking lemma's author's proof of the security of the Schnorr digital signature that they used to demonstrate the use of their lemma in:\\
Mihir Bellare and Gregory Neven - New Multi-Signature Schemes and a General Forking Lemma - Section 4.
\end{reading}

The Bellare and Neven paper of Readings \ref{forking} and \ref{Schnorr} introduced a Schnorr-based multi-signature that was a precursor to MuSig (you can go back and read more of this paper if you wish, but it is probably not the best use of our time together).

\begin{exercise}
In the following reading, pause on page 9 when you reach the SECURITY section, read the first paragraph of that section, and then attempt to define multi-signature scheme security against concurrent attacks using an attack game to capture the notion described in that first paragraph.

Compare your solution to the definition given in the reading.
\end{exercise}

\begin{reading}
Read about MuSig in:\\
Gregory Maxwell, Andrew Poelstra, Yannick Seurin, and Pieter Wuille - Simple Schnorr Multi-Signatures with Applications to Bitcoin.
\end{reading}

As before, on a first reading, I recommend that you don't get too caught up in proof details, but rather return to the proofs once you have gotten a full high-level picture of the protocol and approach.\\

% This brings us to the end of the prework!! You've done it!!!\\

%I will include a few more OPTIONAL readings (in no particular order) below, for those that are interested. I also encourage everyone to take some time to review any material from previous weeks that felt unclear or unresolved.

%\begin{reading}
%Read about MuSig2 in:\\
%Jonas Nick, Tim Ruffing, and Yannick Seurin - MuSig2: Simple Two-Round Schnorr Multi-Signatures.
%\end{reading}

%\begin{reading}
%Read about a commonly-used method of organizing complicated security proofs based on using many indistinguishably small transitions in:\\
%Victor Shoup - Sequences of Games: A Tool for Taming %Complexity in Security Proofs.
%\end{reading}

%In preparation for the next two readings, for any who are interested, I will now provide some exposition on polynomial interpolation. All polynomials in this discussion can be thought of as having coefficients (and hence values) in $\F_p = \Z/p\Z$. This is the mathematical tool underlying most distributed key generation and multi-party computation protocols. At a high level, the idea is to use the fact that a polynomial, $f(x) = y$, of degree $t-1$ is uniquely determined by any $t$ of its points so that we can build $t$-of-$n$ threshold cryptosystems using polynomials to store secrets. For example, a line (degree $1$ polynomial) is determined by any two points and a quadratic polynomial is determined by any three of its points. In this context, rather than individual parties holding entire secrets, they only hold secret shares (or shards), which are actually points on a polynomial (that is, pairs $(x, y)$ satisfying $f(x) = y$). Typically, the secret value $s$ is stored as $f(0) = s$. We can think of representing $f$ as a set of $n$ points $(x_1, y_1), \ldots, (x_n, y_n)$ that each satisfy $f(x_i) = y_i$. If anyone knows at least $t$ of these points (where $f$ is a polynomial of degree $t-1$), then they can reconstruct all of $f$ (as we will discuss below), and in particular, they can compute $f(0) = s$.\\

%One of the reasons to use polynomials (which is known as Shamir secret sharing) in place of, for example, just having everyone XOR secret shares together, is that if we have two shared secrets stored in $f(x) = y$ and $g(x) = y$, then we can add these two secrets together by simply adding corresponding shares together since $(f+g)(x) = f(x) + g(x)$ (where $f+g$ is itself a new polynomial). Similarly, if $c$ is some constant scalar, then $(cf)(x) = c(f(x))$ so that multiplying each share by $c$ results in multiplying the polynomial by $c$. The fact that we can perform addition and scalar multiplication by simply performing operations on secret shares implies that we can compute values such as Schnorr signatures for shared keys without needing to construct the keys!\\

%Polynomial interpolation is the actual process by which one computes a polynomial, $f$, of degree $t-1$ from the input data of $t$ distinct points, $\{(x_i,y_i)\}_{i=1}^t$ satisfying $f(x_i) = y_i$. There are multiple methods of performing interpolation, but one popular method relevant to FROST is Lagrange interpolation. The key idea of Lagrange interpolation is that, given distinct $x_1, x_2,\ldots, x_t$,  the polynomial, $$L_j(x) = \frac{(x-x_1)(x-x_2)\cdots(x-x_{j-1})(x-x_{j+1})\cdots(x-x_t)}{(x_j-x_1)(x_j-x_2)\cdots(x_j-x_{j-1})(x_j-x_{j+1})\cdots(x_j-x_t)} = \prod_{i\neq j}\frac{x-x_i}{x_j - x_i},$$ has the special property that $L_j(x_i)$ is equal to $0$ if $i\neq j$ and is equal to $1$ if $i = j$. Additionally, $L_j(x)$ is a degree $t-1$ polynomial. Therefore, if I wish to build a polynomial that agrees with $f$ on $t$ given points, I can simply take $$L(x) = y_1L_1(x) + y_2L_2(x) + \cdots + y_tL_t(x) = \sum_{j=1}^ty_tL_t(x) = \sum_{j=1}^t\prod_{i\neq j}y_j\frac{x-x_i}{x_j-x_i}.$$ This degree $t-1$ polynomial has the property that $L(x_i) = y_i$ for all $i\in\{1,\ldots, t\}$!\\

%Now that we know how to construct a polynomial that goes through a given $t$ points, we wish to conclude that $L(x) = f(x)$, so we must show that if two degree-at-most $t-1$ polynomials agree on $t$ points, then they must be equal.

%\begin{lemma} If $f$ is a non-zero polynomial such that $f(a) = 0$ (in which case we call $a$ a ``root'' of $f$), then $f(x)$ can be written as $(x-a)\cdot g(x)$ for some polynomial $g(x)$ of degree one less than $f$.
%\end{lemma}
%\begin{proof}
%Expand $f(x+a)$ until it is of the form $c_0 + c_1x + c_2x^2 + \cdots + c_{t-1}x^{t-1}$. Since $f(0+a) = f(a) = 0$, we have that $c_0 = c_0 + c_10 + c_20^2 + \cdots + c_{t-1}0^{t-1} = 0$. Therefore, $f(x+a) = c_1x + c_2x^2 + \cdots + c_{t-1}x^{t-1} = x(c_1 + c_2x + \cdots + c_{t-1}x^{t-2})$, and finally by a change of variables, $f(x) = (x-a)(c_1 + c_2(x-a) + \cdots + c_{t-1}(x-a)^{t-2})$.
%\end{proof}

%\begin{lemma}
%If $f(x)$ is a non-zero degree $t-1$ polynomial, it has at most $t-1$ roots.
%\end{lemma}
%\begin{proof}
%We prove this lemma by induction on $t$. When $t=1$ or $t=2$, the result is clear. Now suppose $f$ has degree $t$ and the result holds for all polynomials of degree less than $t$. If $f$ has no roots then the result is true. If $a$ is a root of $f$, then $f(x) = (x-a)g(x)$ for some polynomial $g(x)$ of degree $t-1$. Therefore, by our inductive hypothesis, $g(x)$ has at most $t-1$ roots so that $f(x)$ has at most $t$ roots, as desired.
%\end{proof}

%\begin{thm} If $L(x)$ and $f(x)$ are two polynomials of degree $t-1$ that agree on $t$ distinct values, then $L(x) = f(x)$.
%\end{thm}
%\begin{proof}
%Consider they polynomial $(L-f)(x) = L(x) - f(x)$, which is of degree at most $t-1$. This polynomial must have at least $t$ distinct roots, because if $L(x_i) = f(x_i)$, then $(L-f)(x_i) = 0$! Therefore, by the previous lemma, $(L-f)(x)$ must be the zero polynomial, so that $(L-f)(x) = L(x) - f(x) = 0\Leftrightarrow L(x) = f(x)$.
%\end{proof}

%At a high level, this theorem is true because Lemma 1 shows us that polynomials can be factored as a product of $(x-a)$'s such that $f(a) = 0$, potentially ending with a factor that has no roots. Thus, a polynomial of degree $t-1$ is determined by any $t$ of its values because the difference of any two polynomials of degree $t-1$ that share $t$ values must be $0$ as it cannot be the product of $t$ linear factors (because the degree is bound by $t-1$).\\

%With this background in mind, we can now read about Pedersen's DKG protocol, wherein $n$ parties can construct a random polynomial known to none of them individually, but where each party knows a point on this polynomial. Subsequently, we can read the FROST paper, which uses this DKG as well as some other Lagrange interpolation tricks to achieve threshold Schnorr digital signatures.

%\begin{reading}
%In preparation for reading about FROST, read about Pedersen's trustless distributed key generation protocol in the extended abstract of:\\
%Torben Pedersen - A Threshold Cryptosystem without a Trusted Party.
%\end{reading}

%\begin{reading}
%Read about FROST in:\\
%Chelsea Komlo and Ian Goldberg - FROST: Flexible Round-Optimized Schnorr Threshold Signatures.
%\end{reading}

\end{document}